\documentclass[11pt,A4]{article}
\usepackage{fullpage}
\usepackage[top=0.5cm, bottom=4cm, left=2cm, right=2cm]{geometry}
\usepackage[utf8]{inputenc}
\usepackage{amsmath,amsthm,amsfonts,amssymb,amscd}
\usepackage{lastpage}
\usepackage{enumerate}
\usepackage{fancyhdr}
\usepackage{mathrsfs}
\usepackage{xcolor}
\usepackage{graphicx}
\usepackage{listings}
\usepackage{hyperref}
\usepackage{pgffor}

\setlength{\parindent}{0.0in}

\pagestyle{fancyplain}
\headheight 35pt
\lhead{\today}
\chead{\textbf{\Large Erkundung des Sonnensystems}}
\rhead{CPT -- Uni Tübingen}
\lfoot{}
\cfoot{}
\rfoot{\footnotesize Thomas.Rometsch@uni-tuebingen.de}
\headsep 1.5em

\begin{document}

\section*{Ziel}

In dieser Aufgabe kannst du einige Eigenschaften des Sonnensystems in einer Simulation beobachten, verstehen und ändern.
Dadurch lernst du mehr über Planetenbahnen, Gravitation und Sonneneinstrahlung.
Gleichzeitig wirst du ein Lab Journal führen und deine Erkenntnisse protokollieren und somit einen Einblick in ein Teilbereich des wissenschaftlichen Arbeitens bekommen.

\section*{Methoden}

Zur Simulation des Sonnensystems wird das Computerprogrammm \textit{Universe Sandbox $^2$} verwendet.
Es berücksichtigt viele verschiedene physikalische Effekte und erlaubt es, die Eigenschaften aller Planeten und Objekte zu ändern, indem das Objekt angeklickt wird und der entsprechende Eintrag angepasst wird.

\section*{Lab Journal}

Dort trägst du alle deine Beobachtungen, Ideen und Gedanken ein, um sie festzuhalten.
Forscher führen meistens ein solches Journal in irgendeiner Art: digital, analog in einem Heft, als Zettelsammlung, \dots
Die Form des Eintrags bleibt jedem selbst überlassen.
Meist besteht ein Eintrag aus mindestens dem Datum und eventuell der Uhrzeit, sowie einem kurzer Titel gefolgt vom Hauptteil..


\subsection*{1. Sonne verschwinden lassen}
Klicke mit der Maus auf die Sonne, sodass auf der rechten Seite des Bildschirms eine Infobox erscheint.
Drücke jetzt die \textit{Entfernen/Delete} Taste, um die Sonne zu löschen.
Beobachte was passiert.
\textit{Notiere die Beobachtung in deinem Lab Journal!}\\\\
Setzte das Sonnensystem zurück, indem du in der oberen linken Ecke auf das Symbol mit den drei Strichen klickst und dann auf \textit{Open} $\to$ \textit{Solar System}.

\subsection*{2. Die Sonne wird zum Schwarzen Loch}
Selektiere wieder die Sonne und klicke auf den zweiten Eintrag (702098 km).
Klicke nocheinmal auf \textit{Radius} und verwende den \textbf{x0.1} Button, um den Radius der Sonne auf ein Paar Kilometer zu verkleinern (Achte darauf, dass die Sonne nicht zu klein wird. Universe Sandbox entfernt sie sonst aus der Simulation!).
\textit{Notiere im Lab Journal was passiert!}

\subsection*{3. Temperatur auf der Erde}
Setzte die Simulation wieder zurück und pausiere die Simulation durch Drücken der Leertaste.
Wiederhole dann Schritt 2, um die Sonne zum Schwarzen Loche zu machen.
Jetzt selektiere die Erde und dann im Eigenschaften Fenster die Temperatur.
Starte die Simulation wieder und beobachte wie sich die Temperatur ändert.
\textit{Notiere im Lab Journal was passiert!}

\subsection*{4. 5. 6. \dots}
Experimentiere weiter und notiere was du beobachtest!

\newpage

\providecommand{\myrule}[1]{\rule{#1}{0.2pt}}
\textit{22.10.19 10:15 : Lab Journal Eintrag\ }\noindent\myrule{0.64\textwidth}\\
\noindent\myrule{0.1\textwidth}\textit{\ Hier kann ich Beobachtungen, Gedanken, Ideen, \dots eintragen.\ }\myrule{0.3\textwidth}\\
\foreach \n in {0,...,45}{\noindent\myrule{\textwidth}\\}


\end{document}
